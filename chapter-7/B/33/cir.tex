\documentclass[journal,10pt,twocolumn]{article}
\usepackage{graphicx, float}
\usepackage[margin=0.5in]{geometry}
\usepackage{amsmath, bm}
\usepackage{array}
\usepackage{booktabs}

\providecommand{\norm}[1]{\left\lVert#1\right\rVert}
\let\vec\mathbf
\newcommand{\myvec}[1]{\ensuremath{\begin{pmatrix}#1\end{pmatrix}}}
\newcommand{\mydet}[1]{\ensuremath{\begin{vmatrix}#1\end{vmatrix}}}

\title{\textbf{Line Assignment}}
\author{Maddu Dinesh}
\date{September 2022}

\begin{document}

\maketitle
\paragraph{\textit{Problem Statement} -Two sides of a rhombus are along the lines x-y+1=0 abd 7x-y-5=0.If diagonals intersect at (-1,-2)then which is the vertex of the rhombus .}

\section*{\large Figure}

\begin{figure}[H]
\centering
\includegraphics[width=1\columnwidth]{line2.png}
\caption{Diagonals intersect at point O(-1,-2)}
\label{fig:triangle}
\end{figure}
\section*{\large Solution}
The vector equation of the line x-y+1=0 AB is
\begin{eqnarray}
	(1\;-1)\myvec{x\\y-1}=0
\end{eqnarray}
The vector equation of the line 7x-y-5=0 AD is
\begin{eqnarray}
	 (7\;-1)\myvec{x\\y+5}=0
\end{eqnarray}



The point of intersection of  eq(1) and eq(2) is
\begin{equation}
\vec{A}=\myvec{1\\2}
\label{eq2}
\end{equation}

The midpoint gives the vertex c
\begin{equation}
\vec{O} = \frac{\vec{A}+\vec{C}}{2}
\label{eq3}
\end{equation}

\begin{equation}
	\vec{C} = \myvec{-3\\-6}
	\label{eq4}
\end{equation}
The slope of AC is the direction vector of AC
\begin{equation}
\vec{m1}=\myvec{4\\8}
\end{equation}


Since AC $\perp$ BD the slope of diagonal BD becomes direction vector
\begin{equation}
\vec{m2}=\myvec{-8\\4}
\end{equation}



Since, AC $\perp$ BD the equation of line in vector along the diagonal BD 
\begin{eqnarray}
	(-8\;4) \myvec{x+1\\y+2}=0
	\label{eq6}
\end{eqnarray}

The point of intersection of eq(8) and eq(1) gives vertex B
\begin{equation}
	\vec{B} = \myvec{-7/3\\-4/3}
	\label{eq7}
\end{equation}
The point of intersection of eq(8) and eq(2) gives vertex D
\begin{equation}
	\vec{D} = \myvec{1/3\\-8/3}
	\label{eq8}
\end{equation}
 Hence the vertices of the rhombus are
 \begin{eqnarray}
	\vec{A} = \myvec{1\\2},
	\vec{B} = \myvec{-7/3\\-4/3},
	\vec{C} = \myvec{-3\\-6},
	\vec{D} = \myvec{1/3\\-8/3}
\end{eqnarray}
 



\section*{\large Construction}
The Vertices of rhombus are
\vspace{6mm}

{
\setlength\extrarowheight{5pt}
\begin{tabular}{|c|c|c|}
	\hline
	\textbf{Symbol}&\textbf{Value}&\textbf{Description}\\
	\hline
	A&(1,2)&vertex at A\\
	\hline
	B&(1/3,-8/3)&vertex at B\\
	\hline
	C&(-3,-6)&vertex at C\\
	\hline
	D&(-7/3,-4/3)&vertex at D\\
	\hline
	O&(-1,-2)& diagonals intersection point\\
	\hline
\end{tabular}
}


\end{document}
